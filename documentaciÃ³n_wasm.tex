\documentclass[a4paper,12pt]{article}

% Paquetes básicos
\usepackage[utf8]{inputenc}
\usepackage[T1]{fontenc}
\usepackage[spanish]{babel}
\usepackage{geometry}
\geometry{margin=2.5cm}

% Paquetes para código
\usepackage{listings}
\usepackage{xcolor}

% Configuración de listings
\lstset{
	basicstyle=\ttfamily\small,
	backgroundcolor=\color[gray]{0.95},
	frame=single,
	breaklines=true,
	showstringspaces=false
}

% Título
\title{Informe de Creación y Ejecución de Proyecto en C y WebAssembly}
\author{Tu Nombre}
\date{\today}

\begin{document}
	
	\maketitle
	
	\section{Introducción}
	Este informe describe el proceso de creación de un programa en C,
	su compilación a WebAssembly mediante \texttt{Emscripten}, la
	configuración de entorno con Vim y Git, así como la ejecución de un
	servidor local para probar los resultados.
	
	\section{Instalación de Emscripten}
	Para instalar Emscripten en Linux (Debian), se siguieron los
	siguientes pasos:
	
	\begin{lstlisting}[language=bash]
		git clone https://github.com/emscripten-core/emsdk.git
		cd emsdk
		./emsdk install latest
		./emsdk activate latest
		source ./emsdk_env.sh
	\end{lstlisting}
	
	\section{Código en C}
	El siguiente programa implementa las funciones de suma y multiplicación:
	
	\begin{lstlisting}[language=C]
		#include <stdio.h>
		
		int sum(int a, int b) {
			return a + b;
		}
		
		int multiply(int a, int b) {
			return a * b;
		}
	\end{lstlisting}
	
	\section{Compilación con Emscripten}
	Se compila el programa a WebAssembly con:
	
	\begin{lstlisting}[language=bash]
		emcc sum.c -o sum.js -s EXPORTED_FUNCTIONS="['_sum','_multiply']"
	\end{lstlisting}
	
	\section{Servidor Local}
	Para activar el servidor en el puerto 8000 y probar el resultado:
	
	\begin{lstlisting}[language=bash]
		python3 -m http.server 8000
	\end{lstlisting}
	
	\section{Plugins de Vim}
	Durante el desarrollo se utilizaron los siguientes plugins de Vim:
	
	\begin{itemize}
		\item \textbf{NERDTree}: exploración de archivos.
		\item \textbf{vim-airline}: barra de estado mejorada.
		\item \textbf{ale / syntastic}: linting en C y HTML.
		\item \textbf{vim-fugitive}: integración con Git.
	\end{itemize}
	
	\section{Uso de Git Flow}
	Para gestionar ramas de desarrollo, se instaló y configuró Git Flow:
	
	\begin{lstlisting}[language=bash]
		sudo apt install git-flow
	\end{lstlisting}
	
	\section{Conclusiones}
	El proyecto permitió integrar C con WebAssembly, probarlo en un
	entorno web sencillo y documentar el flujo de trabajo con herramientas
	modernas de control de versiones y edición.
	
\end{document}

	
	El flujo de ramas incluyó:
	\begin{itemize}
		\item \texttt{main}: versión estable
		\item \texttt{develop}: integración de nuevas funciones
		\item \texttt{feature/*}: ramas de características (ej: multiplicar)
	\end{itemize}
	
	\subsection{Plugins de Vim}
	Se instalaron plugins para mejorar la productividad:
	\begin{itemize}
		\item \textbf{NERDTree}: explorador de archivos
		\item \textbf{vim-airline}: barra de estado mejorada
		\item \textbf{ale}: linting en tiempo real
		\item \textbf{vim-fugitive}: integración con Git
	\end{itemize}
	
	\section{Resultados}
	La aplicación web permite ingresar dos números, realizar operaciones de suma y multiplicación, y visualizar los resultados directamente en la página. El estilo se mejoró con CSS para ofrecer un diseño más agradable en modo oscuro.
	
	\section{Conclusiones}
	Se logró comprender el flujo completo desde la creación del código en C hasta su ejecución en un navegador mediante WebAssembly. Asimismo, se exploraron buenas prácticas en el uso de Git-Flow y Vim como entornos de desarrollo.
	
	\section{Referencias}
	\begin{itemize}
		\item Documentación oficial de Emscripten: \url{https://emscripten.org}
		\item Pandoc para conversión de documentos: \url{https://pandoc.org}
		\item Git-Flow cheatsheet: \url{https://danielkummer.github.io/git-flow-cheatsheet/}
	\end{itemize}
\end{document}
