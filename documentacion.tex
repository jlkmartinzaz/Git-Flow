\documentclass[12pt,a4paper]{article}

% Paquetes básicos
\usepackage[utf8]{inputenc}
\usepackage[spanish]{babel}
\usepackage{listings}
\usepackage{xcolor}   % <-- necesario para listings con backgroundcolor
\usepackage{hyperref}
\usepackage{geometry}
\geometry{margin=2.5cm}

% Estilo de código
\lstset{
	basicstyle=\ttfamily\small,
	backgroundcolor=\color[gray]{0.95},
	frame=single,
	breaklines=true,
	showstringspaces=false,
	tabsize=2
}

\title{Informe Académico: Desarrollo y Compilación con Emscripten}
\author{José Luis Martínez Acevedo}
\date{\today}

\begin{document}
	
	\maketitle
	
	\begin{abstract}
		En este informe se describe el proceso de creación de una aplicación sencilla en C integrada con WebAssembly (WASM), su compilación mediante Emscripten, la visualización en un entorno web y las herramientas adicionales empleadas en el flujo de trabajo, incluyendo Git, Git-Flow y plugins de Vim. 
	\end{abstract}
	
	\section{Introducción}
	El objetivo de este proyecto es comprender cómo se puede integrar código escrito en C con aplicaciones web utilizando WebAssembly, facilitando la ejecución de funciones matemáticas básicas directamente en el navegador.
	
	\section{Metodología}
	A continuación se resumen los pasos principales llevados a cabo:
	
	\subsection{Instalación de Emscripten}
	Se descargó e instaló el SDK de Emscripten:
	\begin{lstlisting}[language=bash]
		git clone https://github.com/emscripten-core/emsdk.git
		cd emsdk
		./emsdk install latest
		./emsdk activate latest
		source ./emsdk_env.sh
	\end{lstlisting}
	
	\subsection{Compilación del código en C}
	El código fuente (\texttt{sum.c}) define dos funciones: suma y multiplicación.
	\begin{lstlisting}[language=C]
		int sum(int a, int b){ return a+b; }
		int multiply(int a, int b){ return a*b; }
	\end{lstlisting}
	
	Se compiló con:
	\begin{lstlisting}[language=bash]
		emcc sum.c -o sum.js -s EXPORTED_FUNCTIONS="['_sum','_multiply']" \
		-s EXPORTED_RUNTIME_METHODS="['ccall','cwrap']"
	\end{lstlisting}
	
	\subsection{Activación de servidor local}
	Para probar la aplicación en el navegador se usó:
	\begin{lstlisting}[language=bash]
		python3 -m http.server 8000
	\end{lstlisting}
	
	\subsection{Uso de Git y Git-Flow}
	Se instaló Git-Flow:
	\begin{lstlisting}[language=bash]
		sudo apt install git-flow
	\end{lstlisting}
	
	El flujo de ramas incluyó:
	\begin{itemize}
		\item \texttt{main}: versión estable.
		\item \texttt{develop}: integración de nuevas funciones.
		\item \texttt{feature/*}: ramas de prueba características en este caso la prueba de poder multiplicar.
	\end{itemize}
	
	\subsection{Prompts empleados}
	Durante el desarrollo se emplearon instrucciones en lenguaje natural para guiar el flujo de trabajo mediante chatgpt o5 basic. Algunos ejemplos son:
	\begin{itemize}
		\item ``Crea un código en C que sume y multiplique dos números.''
		\item ``Explícame cómo compilar con Emscripten y exportar funciones.''
		\item ``Muéstrame cómo levantar un servidor local en Python para probar en el navegador.''
		\item ``Explícame cómo usar Git-Flow para manejar ramas de features.''
		\item ``Recomiéndame plugins de Vim para desarrollo en C y Git.''
	\end{itemize}
	
	\subsection{Plugins de Vim}
	Se instalaron plugins para mejorar la productividad:
	\begin{itemize}
		\item \textbf{NERDTree}: explorador de archivos
		\item \textbf{vim-airline}: barra de estado mejorada
		\item \textbf{ale}: linting en tiempo real
		\item \textbf{vim-fugitive}: integración con Git
	\end{itemize}
	
	\section{Resultados}
	La aplicación web permite ingresar dos números, realizar operaciones de suma y multiplicación, y visualizar los resultados directamente en la página. El estilo se mejoró con CSS para ofrecer un diseño más agradable en modo oscuro.
	
	\section{Conclusiones}
	Se logró comprender el flujo completo desde la creación del código en C hasta su ejecución en un navegador mediante WebAssembly. Asimismo, se exploraron buenas prácticas en el uso de Git-Flow y Vim como entornos de desarrollo.
	
	\section{Referencias}
	\begin{itemize}
		\item Documentación oficial de Emscripten: \url{https://emscripten.org}
		\item Pandoc para conversión de documentos: \url{https://pandoc.org}
		\item Git-Flow cheatsheet: \url{https://danielkummer.github.io/git-flow-cheatsheet/}
	\end{itemize}
	
\end{document}
